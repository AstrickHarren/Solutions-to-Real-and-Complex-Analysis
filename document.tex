\documentclass[10pt,a4paper]{book}
\usepackage[utf8]{inputenc}
\usepackage{amsmath}
\usepackage{amsfonts}
\usepackage{amssymb}
\usepackage{graphicx}
\usepackage[left=2.00cm, right=2.00cm]{geometry}
\author{Astrick}
\date{}
\title{Solution Manual to Real and Complex Analysis}
\usepackage{amsthm}
\theoremstyle{definition}
\newtheorem{exercise}{Exercise}[chapter]
\newcommand{\R}{\mathbb R}
\begin{document}
	\maketitle
	\tableofcontents
	\chapter{Abstract Integration}

	\chapter{Positive Borel Measure}
	\begin{exercise}
		Let $ \{f_n\} $ be a sequence of real non-negative functions on $ \mathbb R^{1} $, and consider the following
		four statements:
		\begin{enumerate}
			\item If $ f_1 $, $ f_2 $ are upper semicontinuous so is $ f_1 + f_2 $.
			\item If $ f_1, f_2 $ are lower semicontinuous, so is $ f_1 + f_2 $.
			\item If each $ f_n $ is upper semicontinuous, so is $ \sum^{\infty} f_n $.
			\item If each $ f_n $ is lower semicontinuous, so is $ \sum^{\infty} f_n $.
		\end{enumerate}
		\paragraph{Solution. }  Observe
		\begin{align*}
		\{f_1 + f_2 < a\} = \bigcup_{x \in \mathbb R} \{f_1 < x\} \cap \{f_2 < a - x\}
		\end{align*}
		is open. To see the left is included in the right, for any $ f_1(y) + f_2(y) < a $, take $ f_1(y) < x < a - f_2(y) $ and the
		inclusion holds. Therefore 1 is verified. Similar argument goes with 2 if the above $ < $ are replaced with $ > $.

		Notice 4 holds, fix any $ x $ such that $ \sum f(x) > a $. Since $ f_n $'s are non-negative, there is $ N $ such that
		$ \sum^{N} f_n(x) > a $. Therefore there exists $\delta$, $ \sum^N f_n(y) > a $ for any $ y \in B_\delta(x) $ since
		finite sums of lower semicontinuous functions are lower semicontinuous. The proof is complete by observing
		\begin{align*}
		\sum f_n(y) \ge \sum^N f_n(y) > a
		\end{align*}

		To give 3 a counterexample, consider $ \sum f_n = \sum \mathcal X_{[-n, -1/n] \cup [1/n, n]} $.
		Obviously, every point but $ 0 $ is greater than or equal to $ 1 $. Hence
		\begin{align*}
		\{\sum f_n < 1 \} = \{0\}
		\end{align*}
		is closed.
	\end{exercise}

	\begin{exercise}
		Let $ f $ be an arbitrary complex function on $ \mathbb R^1 $, and define
		\begin{align*}
		\phi(x, \delta) = \sup \{|f(s) - f(t)|: s, t \in (x - \delta, x + \delta)\}, \\
		\phi(x) = \inf \{\phi(x, \delta): \delta > 0\}.
		\end{align*}
		Prove that $ \phi $ is upper semicontinuous, that $ f $ is continuous at a point $ x $ iff $ \phi(x) = 0 $, and
		hence that the set of points of continuity of an arbitrary complex function is a $ G_{\delta} $.

		Formulate and prove an analogous statement for general topological spaces in place of $ \R^1 $.

		\paragraph{Solution. }
		Only give solution in the general case. Redefine
		\begin{align*}
		\phi(x) = \inf_{B \ni x} \mathrm{diam} f(B)
		\end{align*}
		where the diameter is defined as $ \mathrm{diam} A = \sup_{x, y \in A} |x - y| $. Take any $ x \in \{\phi(x) < a\}$,
		there is $ B \ni x $, $ \mathrm{diam} f(B) < a $. Take any $ y \in B $, then $ \phi(y) \le \mathrm{diam}f(B) < a $.
		This says $  \{\phi(x) < a\} $ is open and $ \phi $ is upper semicontinuous.

		The relation between $ \phi $ and continuity of $ f $ is trivial. Since
		\begin{align*}
		\{\phi = 0\} = \bigcap_{q\in \mathbb Q^+} \{\phi < q\}
		\end{align*}
		the set is a $ G_\delta $.
	\end{exercise}

	\begin{exercise}
		Let $ X $ be a metric space, with metric $ \rho $. For any nonempty $ E \subset X $, define
		\begin{align*}
		\rho_E(x) = \inf_{y \in E} \rho(x, y)
		\end{align*}
		Show that $ \rho_E $ is uniformly continuous function on $ X $. If $ A $ and $ B $ are disjoint nonempty closed
		subsets of $ X $, examine the relevance of the function
		\begin{align*}
		f(x) = \frac{\rho_A(x)}{\rho_A(x) + \rho_B(x)}
		\end{align*}
		to Urysohn's lemma.

		\paragraph{Solution. }
		Notice
		\begin{align*}
		\rho(a, x) + \rho(a, b) \ge \rho(x, b).
		\end{align*}
		Taking infimum on $ E $ on both sides gives
		\begin{align*}
		\rho_E(b) - \rho_E(a) \le \rho(a, b)
		\end{align*}
		By symmetry,
		\begin{align*}
		|\rho_E(b) - \rho_E(a)| \le \rho(a, b)
		\end{align*}
		showing the uniform continuity. Notice $ 0 \le f \le 1 $ and $ f = 1 $ on $ B $. Its support lies in $ A^{c} $.
		Therefore if $ B $ is compact, $ B \prec f \prec A^c $.

	\end{exercise}

	\begin{exercise}
		Examine the proof of Riesz theorem and prove the following two statments:
		\begin{enumerate}
			\item If $ E_1 \subset V_1 $ and $ E_2 \subset V_2 $, where $ V_1 $ and $ V_2 $ are disjoint open sets, then $ \mu(E_1 \cup E_2) = \mu(E_1) + \mu(E_2) $, even if $ E_1, E_2 $ are not in $ \mathcal{M} $.
			\item If $ E \in \mathcal{M}_F $, then $ E = N \cup K_1 \cup K_2 \dots $, where $ \{K_i\} $ is a disjoint countable collection of compact sets and $ \mu(N) = 0 $.
		\end{enumerate}

		\paragraph{Solution. }
		\begin{enumerate}
			\item Take any open set $ U $ that covers $ E_1 \cup E_2 $. Observe
			\begin{align*}
				\mu(U) \ge \mu(U \cap V_1) + \mu(U \cap V_2) \ge \mu(E_1) + \mu(E_2).
			\end{align*}
			Taking infimum on both sides together with the subadditvity of $ \mu $ gives the result.

			\item Since $ E \in \mathcal{M}_F, \mu(E) < \infty $. Take $ K_1 $
			\begin{align*}
				\mu(E) < \mu(K_1) + 1.
			\end{align*}
			Having chosen $ K_1, ..., K_n $, denote $ G = E - \cup ^{n-1} K_i \in \mathcal{M}_F$. Pick $ K_n $ such that
			\begin{align*}
				 \mu(G) < \mu(K_n) + 1/n .
			\end{align*}
		 	Obviously $ \mu(E - \bigcup K_n) = 0 $ and this completes the proof.


		\end{enumerate}
	\end{exercise}

	\begin{exercise}
		Let $ E $ be Cantor's familiar ``middle thirds'' set. Show that $ m(E) = 0 $, even through $ E $ and $ \R ^{1}  $ have the same cardinality.
		\paragraph{Solution. }
		Denote $ R $ as the set removed from $ [0, 1] $ in construction of the Cantor set. Notice it is comprised of the union of $ 2 ^{n-1} $ open intervals of length $ 3 ^{-n} $ where $ n $ ranges in $ \mathbb{N} $. Therefore
		\begin{align*}
			\mu(R) = \sum_{n=1}^{\infty} \frac {2 ^{n-1}}{3^n} = 1
		\end{align*}
		and $ \mu(E) = \mu(I - R) = 0 $.

		To see $ E $ has the same cardinality as $ \R $. Notice each element in $ E $ is a decimal of base 3 that has exactly one 1 at the end or no 1 at all. Therefore there is a surjection from $ \R $ to $ E $, and one from $ E $ to decimals of base 2. This completes the proof.

	\end{exercise}

	\begin{exercise}
		Construct a totally disconnected compact set $ K \subset \R^1 $ such that $ m(K) > 0 $. If $ v $ is lower semicontinuous and $ v \le \mathcal{X}_K $, show that actually $ v \le 0 $. Hence $ \mathcal{X}_K $ cannot be approximated from below by lower semicontinuous functions, in the sense of the Vitali-Carath\'eodory theorem.
		\paragraph{Solution. }
		Construct $ K $ similarly as the Cantor set in the previous exercise only we remove the middle fourths in place of thirds. Let $ R $ be the union of removed intervals. Similarly as above,
		\begin{align*}
			\mu(R) = \sum_{n-1}^{\infty} \frac {2 ^{n-1} }{2 ^{2n} } = \sum \frac {1}{2 ^{n+1} } = \frac {1}{2} .
		\end{align*}
		Therefore $ \mu(K) = 1/2 $. Notice $ \{v > 0\} $ is open by definition. But it cannot be a subset of $ K $ except for the empty set since $ K $ is totally disconnected.

	\end{exercise}

\end{document}
