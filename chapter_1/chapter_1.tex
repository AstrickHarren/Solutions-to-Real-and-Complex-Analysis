\documentclass[../main.tex]{subfiles}
\begin{document}
\chapter{Abstract Integration}

\begin{exercise}
  Does there exist an infinte $ \sigma $-algebra which has only countably many members?
  \paragraph{Solution. }
  If $ \mathcal{F} $ is a $ \sigma $-algebra and $ A $ a set, define $ \mathcal{F} \cap A := \{A \cap F: F \in \mathcal{F}\}$. Hence if $ A \in \mathcal{F} $, then $ \mathcal{F} \cap A $ is a $ \sigma $-subalgebra of $ \mathcal{F} $. Further, if $ \mathcal{F} $ is infinte, either $ A \cap \mathcal{F} $ or $ A^c \cap \mathcal{F}$  is infinte.

  In other words, if $ \mathcal{F}_n $ is infinte, then there is $ A _{n+1} \in \mathcal{F}_n $ such that $ \mathcal{F} _{n+1} = \mathcal{F}_n \cap A_n^c $ is infinte. Take $ \mathcal{F}_0 = \mathcal{M} $, by induction we get a disjoint sequence of sets $ A_n \in \mathcal{M}$.
  Since $ \mathcal{M} $ must also contain any union of sets of $ A_n $, and each union is different, this is to say $ \mathcal{M} $ contains a embedding of all subsets of $ \mathbb{N} $, therefore uncountable.
\end{exercise}

\begin{exercise}
  Prove an analogue of Theorem 1.8 for $ n $ functions.
  \paragraph{Solution. }
  It suffices to prove
  \begin{align*}
    f = f_1 \times f_2 \times ... \times f_n
  \end{align*}
  is measurable is each $ f_i $ is measurable. Similarly as in the theorem, take $ R $ be any rectangles in $ \mathbb{R}^n $. Notice
  \begin{align*}
    f ^{-1}(R) = \bigcap f_i(I_i)
  \end{align*}
  is measurable, if $ R = I_1 \times I_2 \times ... \times I_n $. The rest of the proof is a repeat of Theorem 1.8.
\end{exercise}

\begin{exercise}
  Prove that if $ f$ is a real function on a measurable space $ X $ such that $ \{x: f(x) < r\} $ is measurable for every rational $ r $, then $ f $ is measurable.
  \paragraph{Solution. }
  Take $ \Omega $ as all $ E \subset \mathbb{R} $ such that $ f ^{-1}(E) $ is measurable. By Theorem 1.2, $ \Omega $ is a $ \sigma $-algebra. Notice
  \begin{align*}
    \{f > a\} = \bigcup _{q \in \mathbb{Q} \cap (-\infty, a)} \{f > q\}
  \end{align*}
  for all $ a \in \mathbb{R} $. Therefore, $ \{f > a\} $ is measurable. By Theorem 1.2 again, $ f $ is measurable.

\end{exercise}

\begin{exercise}
  Let $ \{a_n\} $ and $ \{b_n\} $ be sequences in $ [-\infty, \infty] $, and prove the following assertions:
  \begin{enumerate}
    \item $$ \limsup -a_n = -\liminf a_n. $$
    \item $$ \limsup (a_n + b_n) \le \limsup a_n + \limsup b_n $$ provided none of the sums is of the form $ \infty - \infty $.
    \item If $ a_n \le b_n $ for all $ n $, then
    \begin{align*}
      \liminf a_n \le \liminf b_n
    \end{align*}

    Show by example that strict inequality can hold in b).


  \end{enumerate}
  \paragraph{Solution. }
  \begin{enumerate}
    \item First prove $ \sup -A = - \inf A $, where $ A \in \mathbb{\bar R} $. Notice $ -a \le -\inf a $ implies $ \sup -A \le -\inf A $. Similarly $ -\inf A \le \sup -A $.
    \begin{align*}
      \limsup (-a_n) = \inf _{n \ge 1} \sup _{k \ge n} (-a_k) = - \sup \inf a_k = -\limsup a_n
    \end{align*}

    \item Notice if $ \limsup a_n + \limsup b_n $ is defined, $ \sup _{k \ge n} a_k + \sup _{k \ge n} b_k $ is defined for sufficiently large $ n $. Notice
    \begin{align*}
      a_n + b_n \le \sup _{k \ge n} a_k + \sup _{k \ge n} b_k
    \end{align*}
    implies
    \begin{align*}
      \sup _{k \ge n} a_n + b_n \le \sup _{k \ge n} a_k + \sup _{k \ge n} b_k
    \end{align*}
    for sufficiently large $ n $. Taking the limits, whose existence indicated by the question, on both sides gives the result (any decreasing sequence has its limit the same as its infimum).

    \item Clearly $ \inf _{k \ge n} a_k \le \inf _{k \ge n} b_k $ for any $ n $. This completes the proof.
  \end{enumerate}

  Define
  \begin{align*}
    a_n &= 1, -1, 1, -1, 1, ...\\
    b_n &= -1, 1, -1, 1, -1, ...
  \end{align*}
  Obviously, $ \limsup a_n = \limsup b_n = 1 $, while $ \limsup a_n + b_n = 0 $.
\end{exercise}

\begin{exercise}
  \begin{enumerate}
    \item Suppose $ f: X \to [-\infty, \infty] $ and $ g: X \to [-\infty, \infty] $ are measurable. Prove that the sets
    \begin{align*}
      \{x: f(x) < g(x) \}, \{x: f(x) = g(x) \}
    \end{align*}
    are measurable.

    \item Prove that the set of points at which a sequence of measurable real-valued functions converges (to a finite limit) is measurable.

  \end{enumerate}

  \paragraph{Solution. }
  \begin{enumerate}
    \item Notice
    \begin{align*}
      \{f < g\} = \bigcup _{r \in \mathbb Q} \{f < r\} \cap \{r < g\}
    \end{align*}
    therefore measurable. By symmetry, $ \{f > g\} $ is measurable as well. Now $ \{f = g\} $ is the complement of $ \{f > g\} \cup \{f < g\} $ thus must be measurable.

    \item For any sequence of functions $ f_n $, the set of points the sequence converges to a finite limit is
    \begin{align*}
      \{\limsup f_n = \liminf f_n \} \cap \{\limsup f_n < \infty \}.
    \end{align*}
    Since the upper and lower limits of a sequence of functions are measurable functions, the above set is measurable.
  \end{enumerate}

\end{exercise}

\begin{exercise}
  Let $ X $ be an uncountable set, let $ \mathcal{M} $ be the collection of all sets $ E \subset X $ such that either $ E $ or $ E^c $ is at most countable, and define $ \mu(E) = 0 $ in the first case, $ mu(E) = 1 $ in the second. Prove that $ \mathcal{M} $ is a $ \sigma $-algebra in $ X $ and that $ \mu $ is a measure on $ \mathcal{M} $. Describe the corresponding measurable functions and their integrals.

  \paragraph{Solution. }
  It suffices to prove $ \mathcal{M} $ is closed under countable union and $ \mu $ is a measure. Notice for any sequence of sets $ E_n $ in $ \mathcal{M} $, if all of them are at most countable, so must be their union. Therefore $ \bigcup E_n \in \mathcal{M} $, and $ \mu(\bigcup E_n) = 0 = \sum \mu(E_n) $. If $ E_i $ have its complement at most countable, $ \left (\bigcup E_n \right )^c = \bigcap E_n^c \subset E_i^c$ is at most countable. Hence $ \bigcup E_n \in \mathcal{M} $.
  Further, if $ E_n $ are disjoint, for any set $ E_j, j \ne i $, $ X \subset E_i^c \cup E_j^c $, showing $ E_j^c $ must be uncountable, therefore $ E_j $ must be at most countable and $ \sum \mu(E_n) = 1 = \mu \left (\bigcup E_n \right ) $.

  Only describe integrals of non-negative functions, the general case follows is of no interest. Since any singalton is Borel, for any $ r \in [0, \infty] $, a measurable function $ f: X \to [0, \infty] $ would have $ f ^{-1}(r) $ either at most countable or have a at most countable complement. Take $ A $ as the set of $ r $ such that $ f ^{-1}(r)  $ has an at most countable complement. Then if $ A $ is countable, it is easy to check $ \int f = \sum _{r \in A} r  $. If $ A $ is uncountable, then $ \int f = \infty $. 21


\end{exercise}


\end{document}
